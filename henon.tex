\documentclass[twocolumn]{article}
\usepackage{amsmath,amsfonts}
\usepackage{amsmath,amsfonts}
%\usepackage{showlabels}
\usepackage[pdftex]{graphicx,color}
\title{Using the Henon System to Illustrate Software Tools}

\author{Andrew M.\ Fraser}
\begin{document}
\maketitle
\begin{abstract}
  This document illustrates most of the results of the examples I will
  use on the first day of lectures on software tools at LANL in June
  of 2013.  One should be able to build this document by fetching the
  source from https://github.com/fraserphysics/python13 and typing
  \emph{make} at a Linux command line.
\end{abstract}

\section{From Wikipedia}
\label{sec:wikipedia}

From http://en.wikipedia.org/wiki/Henon\_map, I copied this equation
\begin{align*}
   x_{n+1} &= y_n+1-a x_n^2,\\
  y_{n+1} &= b x_n
\end{align*}
and the values $a = 1.4$ and $b = 0.3$.

\section{Figures}
\label{sec:figures}

One can use gnu-make, \texttt{Makefile} and the code in \texttt{plot.py} and
\texttt{henon.py} to make Fig.~\ref{fig:dots}
\begin{figure}
  \centering
  \resizebox{0.95\columnwidth}{!}{\includegraphics{dots.pdf}}
  \caption{Five thousand iterations of the Henon system.}
  \label{fig:dots}
\end{figure}

Figure \ref{fig:colors} illustrates the varying density at finite
resolution.
\begin{figure}
  \centering
  \resizebox{0.95\columnwidth}{!}{\includegraphics{colors.pdf}}
  \caption{The colors indicate the count in each box from a
    trajectory of length 10,000.  The box
    resolution is $\Delta_x = 0.009$, $\Delta_y=0.003$. }
  \label{fig:colors}
\end{figure}

Figure \ref{fig:dim} is a \emph{blast from the past}.  In the 1980's
and 1990's the \emph{dimension industry} filled shelves in libraries
with plots of $\log(N)$ as a function of $\log(\Delta_x)$.  To
demonstrate BibTeX I quote the citation\cite{russell1980} that
Wikipedia uses for the Hausdorff dimension of the Henon attractor.
\begin{figure}
  \centering
  \resizebox{0.95\columnwidth}{!}{\includegraphics{dim.pdf}}
  \caption{Log of the number of occupied boxes as a function of the log
    of the box edge length using sample trajectory of 10,000 samples.
    Fit a line to the sloping straight part of this plot to estimate
    \emph{the box counting dimension} of the attractor.  To get more
    range, you could use python's \emph{set} data type.}
  \label{fig:dim}
\end{figure}

Figure \ref{fig:count} presents the the log of the number of samples
that fall in each box plotted against the log of the rank of the box.
Perhaps the constant slope of the part of the curve on the left
indicates a \emph{power law}.
\begin{figure}
  \centering
  \resizebox{0.95\columnwidth}{!}{\includegraphics{count.pdf}}
  \caption{Log of the number of samples in each occupied box plotted
    against the log of the box rank.  The trajectory analyzed has
    100,000 samples and the $x$ and $y$ resolutions are 0.009 and
    0.003 respectively.  Using python's \emph{dict} data type might
    improve the performance of the code.}
  \label{fig:count}
\end{figure}

\section{Exercises}
\label{sec:exercises}
\begin{enumerate}
\item \label{it:set} Make an version of Fig.~\ref{fig:dim} that
  extends the x-axis to -8 by writing your own versions of the
  functions \texttt{bins} and \texttt{count\_array} in the file
  \texttt{henon.py}.  I believe that by using python's \emph{set} data
  type and by binning each sample at all resolutions before going to
  the next iteration, you can get both speed and memory performance
  improvements over my code.
\item Make an version of Fig.~\ref{fig:count} by writing your own
  versions of the functions \texttt{bins}, \texttt{count\_array} and
  \texttt{log\_log} in the file \texttt{henon.py}.  Use the python
  \emph{dict} data type instead of numpy arrays for counting the
  number of samples in each occupied box.  I believe that you will get
  improved memory performance but not much speed improvement.
\end{enumerate}

\bibliographystyle{unsrt} \bibliography{local}
\end{document}

%%%---------------
%%% Local Variables:
%%% eval: (TeX-PDF-mode)
%%% End:
